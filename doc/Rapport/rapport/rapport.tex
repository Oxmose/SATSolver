\documentclass{article}

\usepackage[utf8]{inputenc}
\usepackage[french]{babel}

% elements du titre
\title{Rapport projet 2, TSAT}
\author{T. Stérin, A. Torres}
\date{Mai 2016}


% definition de quelques macros, pour les maths
\newcommand{\litt}{\alpha}
\newcommand{\non}[1]{\overline{#1}}

\begin{document}

\maketitle

\section{Présentation}
TSAT est un solveur SAT écrit en C++ par Tristan Stérin et Alexy Torres--Aurora-Dugo.
Réalisé a l'ENS Lyon pour l'année scolaire 2015-2016 et dans le cadre de la matière ``Projet2'', ce solveur permet de résoudre des formules SAT sous formes CNF et logique (en appliquant la transformation de Tseitin).
\\



Un peu de maths en \LaTeX: voici un exemple de clause:
$$
\litt_1\lor\non{\litt_2}\lor\litt_4
$$
On remarque au passage que $\non{\non{\litt}}$ est pareil que $\litt$.

\section{Organisation du code}
\label{s:orga}

Le code est structuré de la manière suivante~:
\begin{itemize}
\item bli
\item bla
\item blo
\end{itemize}

\section{Critique des performances}

On constate que blibla.


On est par ailleurs capable de citer des références, ainsi~: \cite{ProjInt16}.


Pour citer une référence bibliographique, il faut insérer les
informations correspondantes au format BibTeX dans le fichier
\texttt{ex-biblio.bib}, et puis faire la citation en utilisant la
commande \verb+\cite{tititoto}+.

Ensuite, on compile de la manière suivante~:
\begin{enumerate}
\item \texttt{pdflatex ex-rapport}

et là il proteste, car il a vu une citation de \texttt{tititoto}, mais
ne sait pas à quoi cela fait référence

\item \texttt{bibtex ex-rapport}

et là il met ensemble les informations pour savoir engendrer
l'information correspondant à la citation de \texttt{tititoto}

\item \texttt{pdflatex ex-rapport}

et là il peut engendrer le fichier pdf, avec la bonne citation et la
bonne description dans les références
\end{enumerate}


\bibliographystyle{plain}
\bibliography{ex-biblio}

\end{document}
